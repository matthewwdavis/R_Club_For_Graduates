% Options for packages loaded elsewhere
\PassOptionsToPackage{unicode}{hyperref}
\PassOptionsToPackage{hyphens}{url}
%
\documentclass[
]{article}
\usepackage{amsmath,amssymb}
\usepackage{lmodern}
\usepackage{iftex}
\ifPDFTeX
  \usepackage[T1]{fontenc}
  \usepackage[utf8]{inputenc}
  \usepackage{textcomp} % provide euro and other symbols
\else % if luatex or xetex
  \usepackage{unicode-math}
  \defaultfontfeatures{Scale=MatchLowercase}
  \defaultfontfeatures[\rmfamily]{Ligatures=TeX,Scale=1}
\fi
% Use upquote if available, for straight quotes in verbatim environments
\IfFileExists{upquote.sty}{\usepackage{upquote}}{}
\IfFileExists{microtype.sty}{% use microtype if available
  \usepackage[]{microtype}
  \UseMicrotypeSet[protrusion]{basicmath} % disable protrusion for tt fonts
}{}
\makeatletter
\@ifundefined{KOMAClassName}{% if non-KOMA class
  \IfFileExists{parskip.sty}{%
    \usepackage{parskip}
  }{% else
    \setlength{\parindent}{0pt}
    \setlength{\parskip}{6pt plus 2pt minus 1pt}}
}{% if KOMA class
  \KOMAoptions{parskip=half}}
\makeatother
\usepackage{xcolor}
\usepackage[margin=1in]{geometry}
\usepackage{color}
\usepackage{fancyvrb}
\newcommand{\VerbBar}{|}
\newcommand{\VERB}{\Verb[commandchars=\\\{\}]}
\DefineVerbatimEnvironment{Highlighting}{Verbatim}{commandchars=\\\{\}}
% Add ',fontsize=\small' for more characters per line
\usepackage{framed}
\definecolor{shadecolor}{RGB}{248,248,248}
\newenvironment{Shaded}{\begin{snugshade}}{\end{snugshade}}
\newcommand{\AlertTok}[1]{\textcolor[rgb]{0.94,0.16,0.16}{#1}}
\newcommand{\AnnotationTok}[1]{\textcolor[rgb]{0.56,0.35,0.01}{\textbf{\textit{#1}}}}
\newcommand{\AttributeTok}[1]{\textcolor[rgb]{0.77,0.63,0.00}{#1}}
\newcommand{\BaseNTok}[1]{\textcolor[rgb]{0.00,0.00,0.81}{#1}}
\newcommand{\BuiltInTok}[1]{#1}
\newcommand{\CharTok}[1]{\textcolor[rgb]{0.31,0.60,0.02}{#1}}
\newcommand{\CommentTok}[1]{\textcolor[rgb]{0.56,0.35,0.01}{\textit{#1}}}
\newcommand{\CommentVarTok}[1]{\textcolor[rgb]{0.56,0.35,0.01}{\textbf{\textit{#1}}}}
\newcommand{\ConstantTok}[1]{\textcolor[rgb]{0.00,0.00,0.00}{#1}}
\newcommand{\ControlFlowTok}[1]{\textcolor[rgb]{0.13,0.29,0.53}{\textbf{#1}}}
\newcommand{\DataTypeTok}[1]{\textcolor[rgb]{0.13,0.29,0.53}{#1}}
\newcommand{\DecValTok}[1]{\textcolor[rgb]{0.00,0.00,0.81}{#1}}
\newcommand{\DocumentationTok}[1]{\textcolor[rgb]{0.56,0.35,0.01}{\textbf{\textit{#1}}}}
\newcommand{\ErrorTok}[1]{\textcolor[rgb]{0.64,0.00,0.00}{\textbf{#1}}}
\newcommand{\ExtensionTok}[1]{#1}
\newcommand{\FloatTok}[1]{\textcolor[rgb]{0.00,0.00,0.81}{#1}}
\newcommand{\FunctionTok}[1]{\textcolor[rgb]{0.00,0.00,0.00}{#1}}
\newcommand{\ImportTok}[1]{#1}
\newcommand{\InformationTok}[1]{\textcolor[rgb]{0.56,0.35,0.01}{\textbf{\textit{#1}}}}
\newcommand{\KeywordTok}[1]{\textcolor[rgb]{0.13,0.29,0.53}{\textbf{#1}}}
\newcommand{\NormalTok}[1]{#1}
\newcommand{\OperatorTok}[1]{\textcolor[rgb]{0.81,0.36,0.00}{\textbf{#1}}}
\newcommand{\OtherTok}[1]{\textcolor[rgb]{0.56,0.35,0.01}{#1}}
\newcommand{\PreprocessorTok}[1]{\textcolor[rgb]{0.56,0.35,0.01}{\textit{#1}}}
\newcommand{\RegionMarkerTok}[1]{#1}
\newcommand{\SpecialCharTok}[1]{\textcolor[rgb]{0.00,0.00,0.00}{#1}}
\newcommand{\SpecialStringTok}[1]{\textcolor[rgb]{0.31,0.60,0.02}{#1}}
\newcommand{\StringTok}[1]{\textcolor[rgb]{0.31,0.60,0.02}{#1}}
\newcommand{\VariableTok}[1]{\textcolor[rgb]{0.00,0.00,0.00}{#1}}
\newcommand{\VerbatimStringTok}[1]{\textcolor[rgb]{0.31,0.60,0.02}{#1}}
\newcommand{\WarningTok}[1]{\textcolor[rgb]{0.56,0.35,0.01}{\textbf{\textit{#1}}}}
\usepackage{graphicx}
\makeatletter
\def\maxwidth{\ifdim\Gin@nat@width>\linewidth\linewidth\else\Gin@nat@width\fi}
\def\maxheight{\ifdim\Gin@nat@height>\textheight\textheight\else\Gin@nat@height\fi}
\makeatother
% Scale images if necessary, so that they will not overflow the page
% margins by default, and it is still possible to overwrite the defaults
% using explicit options in \includegraphics[width, height, ...]{}
\setkeys{Gin}{width=\maxwidth,height=\maxheight,keepaspectratio}
% Set default figure placement to htbp
\makeatletter
\def\fps@figure{htbp}
\makeatother
\setlength{\emergencystretch}{3em} % prevent overfull lines
\providecommand{\tightlist}{%
  \setlength{\itemsep}{0pt}\setlength{\parskip}{0pt}}
\setcounter{secnumdepth}{-\maxdimen} % remove section numbering
\ifLuaTeX
  \usepackage{selnolig}  % disable illegal ligatures
\fi
\IfFileExists{bookmark.sty}{\usepackage{bookmark}}{\usepackage{hyperref}}
\IfFileExists{xurl.sty}{\usepackage{xurl}}{} % add URL line breaks if available
\urlstyle{same} % disable monospaced font for URLs
\hypersetup{
  pdftitle={2022\_08\_29\_chapter2},
  pdfauthor={Matthew Davis},
  hidelinks,
  pdfcreator={LaTeX via pandoc}}

\title{2022\_08\_29\_chapter2}
\author{Matthew Davis}
\date{2022-08-28}

\begin{document}
\maketitle

\begin{Shaded}
\begin{Highlighting}[]
\FunctionTok{library}\NormalTok{(tidyverse)}
\end{Highlighting}
\end{Shaded}

\begin{verbatim}
## -- Attaching packages --------------------------------------- tidyverse 1.3.2 --
## v ggplot2 3.3.6     v purrr   0.3.4
## v tibble  3.1.8     v dplyr   1.0.9
## v tidyr   1.2.0     v stringr 1.4.0
## v readr   2.1.2     v forcats 0.5.1
## -- Conflicts ------------------------------------------ tidyverse_conflicts() --
## x dplyr::filter() masks stats::filter()
## x dplyr::lag()    masks stats::lag()
\end{verbatim}

\begin{Shaded}
\begin{Highlighting}[]
\FunctionTok{library}\NormalTok{(lobstr)}
\end{Highlighting}
\end{Shaded}

\hypertarget{chapter-2-names-and-values}{%
\subsection{Chapter 2: Names and
Values}\label{chapter-2-names-and-values}}

Quiz 1. Given the following data frame, how do I create a new column
called ``3'' that contains the sum of \texttt{1} and \texttt{2}? You may
only use \texttt{\$}, not \texttt{{[}{[}}. What makes \texttt{1},
\texttt{2}, and \texttt{3} challenging as variable names?

\begin{Shaded}
\begin{Highlighting}[]
\CommentTok{\# Dataframe}
\NormalTok{df }\OtherTok{\textless{}{-}} \FunctionTok{data.frame}\NormalTok{(}\FunctionTok{runif}\NormalTok{(}\DecValTok{3}\NormalTok{), }\FunctionTok{runif}\NormalTok{(}\DecValTok{3}\NormalTok{))}
\FunctionTok{names}\NormalTok{(df) }\OtherTok{\textless{}{-}} \FunctionTok{c}\NormalTok{(}\DecValTok{1}\NormalTok{, }\DecValTok{2}\NormalTok{)}
\NormalTok{df}
\end{Highlighting}
\end{Shaded}

\begin{verbatim}
##           1         2
## 1 0.8825320 0.6600093
## 2 0.3907382 0.4431045
## 3 0.5044674 0.2879284
\end{verbatim}

\begin{Shaded}
\begin{Highlighting}[]
\CommentTok{\# Solution}
\NormalTok{df }\SpecialCharTok{\%\textgreater{}\%}
  \FunctionTok{mutate}\NormalTok{(}\StringTok{\textasciigrave{}}\AttributeTok{3}\StringTok{\textasciigrave{}} \OtherTok{=} \StringTok{\textasciigrave{}}\AttributeTok{1}\StringTok{\textasciigrave{}}\SpecialCharTok{+}\StringTok{\textasciigrave{}}\AttributeTok{2}\StringTok{\textasciigrave{}}\NormalTok{)}
\end{Highlighting}
\end{Shaded}

\begin{verbatim}
##           1         2         3
## 1 0.8825320 0.6600093 1.5425413
## 2 0.3907382 0.4431045 0.8338427
## 3 0.5044674 0.2879284 0.7923958
\end{verbatim}

\begin{enumerate}
\def\labelenumi{\arabic{enumi}.}
\setcounter{enumi}{1}
\tightlist
\item
  In the following code, how much memory does \texttt{y} occupy?
\end{enumerate}

\begin{Shaded}
\begin{Highlighting}[]
\NormalTok{x }\OtherTok{\textless{}{-}} \FunctionTok{runif}\NormalTok{(}\FloatTok{1e6}\NormalTok{)}
\NormalTok{y }\OtherTok{\textless{}{-}} \FunctionTok{list}\NormalTok{(x, x, x) }\CommentTok{\# 8Mb}
\end{Highlighting}
\end{Shaded}

\begin{enumerate}
\def\labelenumi{\arabic{enumi}.}
\setcounter{enumi}{2}
\tightlist
\item
  On which line does \texttt{a} get copied in the following example?
\end{enumerate}

\begin{Shaded}
\begin{Highlighting}[]
\NormalTok{a }\OtherTok{\textless{}{-}} \FunctionTok{c}\NormalTok{(}\DecValTok{1}\NormalTok{, }\DecValTok{5}\NormalTok{, }\DecValTok{3}\NormalTok{, }\DecValTok{2}\NormalTok{)}
\NormalTok{b }\OtherTok{\textless{}{-}}\NormalTok{ a}
\NormalTok{b[[}\DecValTok{1}\NormalTok{]] }\OtherTok{\textless{}{-}} \DecValTok{10}
\NormalTok{b}
\end{Highlighting}
\end{Shaded}

\begin{verbatim}
## [1] 10  5  3  2
\end{verbatim}

\begin{Shaded}
\begin{Highlighting}[]
\CommentTok{\# on the first line}
\end{Highlighting}
\end{Shaded}

Acccessing an objects identifiers

\begin{Shaded}
\begin{Highlighting}[]
\FunctionTok{obj\_addr}\NormalTok{(df)}
\end{Highlighting}
\end{Shaded}

\begin{verbatim}
## [1] "0x124258108"
\end{verbatim}

\begin{Shaded}
\begin{Highlighting}[]
\FunctionTok{obj\_addr}\NormalTok{(y)}
\end{Highlighting}
\end{Shaded}

\begin{verbatim}
## [1] "0x122efcff8"
\end{verbatim}

Identifiers change every time R is restarted

When creating names, use backticks instead of '' or ' for consistency of
naming context

\textbf{2.2.2 Exercises}

\begin{enumerate}
\def\labelenumi{\arabic{enumi}.}
\tightlist
\item
  Explain the relationship between a, b, c and d in the following code:
\end{enumerate}

\begin{Shaded}
\begin{Highlighting}[]
\NormalTok{a }\OtherTok{\textless{}{-}} \DecValTok{1}\SpecialCharTok{:}\DecValTok{10}
\NormalTok{b }\OtherTok{\textless{}{-}}\NormalTok{ a}
\NormalTok{c }\OtherTok{\textless{}{-}}\NormalTok{ b}
\NormalTok{d }\OtherTok{\textless{}{-}} \DecValTok{1}\SpecialCharTok{:}\DecValTok{10}
\end{Highlighting}
\end{Shaded}

a is b and c.~d contains the same values, but was created without
association with a, b, and c

\begin{enumerate}
\def\labelenumi{\arabic{enumi}.}
\setcounter{enumi}{1}
\tightlist
\item
  The following code accesses the mean function in multiple ways. Do
  they all point to the same underlying function object? Verify this
  with \texttt{lobstr::obj\_addr()}.
\end{enumerate}

\begin{Shaded}
\begin{Highlighting}[]
\NormalTok{mean}
\end{Highlighting}
\end{Shaded}

\begin{verbatim}
## function (x, ...) 
## UseMethod("mean")
## <bytecode: 0x1200d13a8>
## <environment: namespace:base>
\end{verbatim}

\begin{Shaded}
\begin{Highlighting}[]
\NormalTok{base}\SpecialCharTok{::}\NormalTok{mean}
\end{Highlighting}
\end{Shaded}

\begin{verbatim}
## function (x, ...) 
## UseMethod("mean")
## <bytecode: 0x1200d13a8>
## <environment: namespace:base>
\end{verbatim}

\begin{Shaded}
\begin{Highlighting}[]
\FunctionTok{get}\NormalTok{(}\StringTok{"mean"}\NormalTok{)}
\end{Highlighting}
\end{Shaded}

\begin{verbatim}
## function (x, ...) 
## UseMethod("mean")
## <bytecode: 0x1200d13a8>
## <environment: namespace:base>
\end{verbatim}

\begin{Shaded}
\begin{Highlighting}[]
\FunctionTok{evalq}\NormalTok{(mean)}
\end{Highlighting}
\end{Shaded}

\begin{verbatim}
## function (x, ...) 
## UseMethod("mean")
## <bytecode: 0x1200d13a8>
## <environment: namespace:base>
\end{verbatim}

\begin{Shaded}
\begin{Highlighting}[]
\FunctionTok{match.fun}\NormalTok{(}\StringTok{"mean"}\NormalTok{)}
\end{Highlighting}
\end{Shaded}

\begin{verbatim}
## function (x, ...) 
## UseMethod("mean")
## <bytecode: 0x1200d13a8>
## <environment: namespace:base>
\end{verbatim}

In Markdown, the addresses are told to you, so it seems you don't need
\texttt{obj\_addr()} to confirm the object id

\begin{enumerate}
\def\labelenumi{\arabic{enumi}.}
\setcounter{enumi}{2}
\tightlist
\item
  By default, base R data import functions, like \texttt{read.csv()},
  will automatically convert non-syntactic names to syntactic ones. Why
  might this be problematic? What option allows you to suppress this
  behaviour?
\end{enumerate}

\begin{Shaded}
\begin{Highlighting}[]
\NormalTok{?read.csv}
\CommentTok{\# Use \textasciigrave{}check.names = FALSE\textasciigrave{}}
\end{Highlighting}
\end{Shaded}

Changing the names could create uncertainty in what is being displayed

\begin{enumerate}
\def\labelenumi{\arabic{enumi}.}
\setcounter{enumi}{3}
\item
  What rules does \texttt{make.names()} use to convert non-syntactic
  names into syntactic ones?
  \texttt{make.names(names,\ unique\ =\ FALSE,\ allow\_\ =\ TRUE)}
\item
  I slightly simplified the rules that govern syntactic names. Why is
  \texttt{.123e1} not a syntactic name? Read \texttt{?make.names} for
  the full details It could be due to invalid characters being converted
  to \texttt{.}
\end{enumerate}

See when an object gets coppied with \texttt{tracemem()}

\begin{Shaded}
\begin{Highlighting}[]
\NormalTok{x }\OtherTok{\textless{}{-}} \FunctionTok{c}\NormalTok{(}\DecValTok{1}\NormalTok{, }\DecValTok{2}\NormalTok{, }\DecValTok{3}\NormalTok{)}
\FunctionTok{cat}\NormalTok{(}\FunctionTok{tracemem}\NormalTok{(x), }\StringTok{"}\SpecialCharTok{\textbackslash{}n}\StringTok{"}\NormalTok{)}
\end{Highlighting}
\end{Shaded}

\begin{verbatim}
## <0x1203b9678>
\end{verbatim}

\begin{Shaded}
\begin{Highlighting}[]
\CommentTok{\#From here on, whenever that object is copied, \textasciigrave{}tracemem()\textasciigrave{} will print a message telling you which object was copied, its new address, and the sequence of calls that led to the copy:}

\NormalTok{y }\OtherTok{\textless{}{-}}\NormalTok{ x}
\NormalTok{y[[}\DecValTok{3}\NormalTok{]] }\OtherTok{\textless{}{-}} \DecValTok{4}
\end{Highlighting}
\end{Shaded}

\begin{verbatim}
## tracemem[0x1203b9678 -> 0x1203a2628]: eval eval eval_with_user_handlers withVisible withCallingHandlers handle timing_fn evaluate_call <Anonymous> evaluate in_dir in_input_dir eng_r block_exec call_block process_group.block process_group withCallingHandlers process_file <Anonymous> <Anonymous>
\end{verbatim}

\begin{Shaded}
\begin{Highlighting}[]
\CommentTok{\# \textasciigrave{}untracemem()\textasciigrave{} turns tracing off}
\FunctionTok{untracemem}\NormalTok{(x)}
\end{Highlighting}
\end{Shaded}

R uses refrences with character vectors

\begin{Shaded}
\begin{Highlighting}[]
\NormalTok{x }\OtherTok{\textless{}{-}} \FunctionTok{c}\NormalTok{(}\StringTok{"a"}\NormalTok{, }\StringTok{"a"}\NormalTok{, }\StringTok{"abc"}\NormalTok{, }\StringTok{"d"}\NormalTok{)}

\CommentTok{\# \textasciigrave{}ref()\textasciigrave{} shows the references r uses when \textasciigrave{}character = TRUE\textasciigrave{}}

\FunctionTok{ref}\NormalTok{(x, }\AttributeTok{character =} \ConstantTok{TRUE}\NormalTok{)}
\end{Highlighting}
\end{Shaded}

\begin{verbatim}
## █ [1:0x1241cc838] <chr> 
## ├─[2:0x147051a08] <string: "a"> 
## ├─[2:0x147051a08] 
## ├─[3:0x124075b98] <string: "abc"> 
## └─[4:0x147200bb0] <string: "d">
\end{verbatim}

2.3.6 Exercises 1. Why is \texttt{tracemem(1:10)} not useful?

\begin{Shaded}
\begin{Highlighting}[]
\FunctionTok{tracemem}\NormalTok{(}\DecValTok{1}\SpecialCharTok{:}\DecValTok{10}\NormalTok{)}
\end{Highlighting}
\end{Shaded}

\begin{verbatim}
## [1] "<0x123043600>"
\end{verbatim}

There was no object created, so the ID is not going to be consistent.

\begin{enumerate}
\def\labelenumi{\arabic{enumi}.}
\setcounter{enumi}{1}
\tightlist
\item
  Explain why \texttt{tracemem()} shows two copies when you run this
  code. Hint: carefully look at the difference between this code and the
  code shown earlier in the section
\end{enumerate}

\begin{Shaded}
\begin{Highlighting}[]
\NormalTok{x }\OtherTok{\textless{}{-}} \FunctionTok{c}\NormalTok{(1L, 2L, 3L)}
\FunctionTok{tracemem}\NormalTok{(x)}
\end{Highlighting}
\end{Shaded}

\begin{verbatim}
## [1] "<0x124124dc8>"
\end{verbatim}

\begin{Shaded}
\begin{Highlighting}[]
\NormalTok{x[[}\DecValTok{3}\NormalTok{]] }\OtherTok{\textless{}{-}} \DecValTok{4}
\end{Highlighting}
\end{Shaded}

\begin{verbatim}
## tracemem[0x124124dc8 -> 0x1240e0708]: eval eval eval_with_user_handlers withVisible withCallingHandlers handle timing_fn evaluate_call <Anonymous> evaluate in_dir in_input_dir eng_r block_exec call_block process_group.block process_group withCallingHandlers process_file <Anonymous> <Anonymous> 
## tracemem[0x1240e0708 -> 0x1401526a8]: eval eval eval_with_user_handlers withVisible withCallingHandlers handle timing_fn evaluate_call <Anonymous> evaluate in_dir in_input_dir eng_r block_exec call_block process_group.block process_group withCallingHandlers process_file <Anonymous> <Anonymous>
\end{verbatim}

Because the new value is non-syntactic. Is it being saved as two
classes?

\begin{enumerate}
\def\labelenumi{\arabic{enumi}.}
\setcounter{enumi}{2}
\tightlist
\item
  Sketch out the relationship between the following objects:
\end{enumerate}

\begin{Shaded}
\begin{Highlighting}[]
\NormalTok{a }\OtherTok{\textless{}{-}} \DecValTok{1}\SpecialCharTok{:}\DecValTok{10}

\FunctionTok{tracemem}\NormalTok{(a)}
\end{Highlighting}
\end{Shaded}

\begin{verbatim}
## [1] "<0x1222d8978>"
\end{verbatim}

\begin{Shaded}
\begin{Highlighting}[]
\NormalTok{b }\OtherTok{\textless{}{-}} \FunctionTok{list}\NormalTok{(a, a)}

\FunctionTok{tracemem}\NormalTok{(b)}
\end{Highlighting}
\end{Shaded}

\begin{verbatim}
## [1] "<0x122eb2a48>"
\end{verbatim}

\begin{Shaded}
\begin{Highlighting}[]
\NormalTok{c }\OtherTok{\textless{}{-}} \FunctionTok{list}\NormalTok{(b, a, }\DecValTok{1}\SpecialCharTok{:}\DecValTok{10}\NormalTok{)}

\FunctionTok{tracemem}\NormalTok{(c)}
\end{Highlighting}
\end{Shaded}

\begin{verbatim}
## [1] "<0x124246fd8>"
\end{verbatim}

\begin{enumerate}
\def\labelenumi{\arabic{enumi}.}
\setcounter{enumi}{3}
\tightlist
\item
  What happens when you run this code?
\end{enumerate}

\begin{Shaded}
\begin{Highlighting}[]
\NormalTok{x }\OtherTok{\textless{}{-}} \FunctionTok{list}\NormalTok{(}\DecValTok{1}\SpecialCharTok{:}\DecValTok{10}\NormalTok{)}
\NormalTok{x[[}\DecValTok{2}\NormalTok{]] }\OtherTok{\textless{}{-}}\NormalTok{ x}
\NormalTok{x}
\end{Highlighting}
\end{Shaded}

\begin{verbatim}
## [[1]]
##  [1]  1  2  3  4  5  6  7  8  9 10
## 
## [[2]]
## [[2]][[1]]
##  [1]  1  2  3  4  5  6  7  8  9 10
\end{verbatim}

2.4.1 Exercises

\begin{enumerate}
\def\labelenumi{\arabic{enumi}.}
\tightlist
\item
  In the following example, why are \texttt{object.size(y)} and
  \texttt{obj\_size(y)} so radically different? Consult the
  documentation of \texttt{object.size()}
\end{enumerate}

\begin{Shaded}
\begin{Highlighting}[]
\NormalTok{y }\OtherTok{\textless{}{-}} \FunctionTok{rep}\NormalTok{(}\FunctionTok{list}\NormalTok{(}\FunctionTok{runif}\NormalTok{(}\FloatTok{1e4}\NormalTok{)), }\DecValTok{100}\NormalTok{)}

\FunctionTok{object.size}\NormalTok{(y)}
\end{Highlighting}
\end{Shaded}

\begin{verbatim}
## 8005648 bytes
\end{verbatim}

\begin{Shaded}
\begin{Highlighting}[]
\FunctionTok{obj\_size}\NormalTok{(y)}
\end{Highlighting}
\end{Shaded}

\begin{verbatim}
## 80.90 kB
\end{verbatim}

\begin{Shaded}
\begin{Highlighting}[]
\NormalTok{?object.size}
\end{Highlighting}
\end{Shaded}

\texttt{obj\_size} is reporting in kB, \texttt{object.size} is reporting
in B

\begin{enumerate}
\def\labelenumi{\arabic{enumi}.}
\setcounter{enumi}{1}
\tightlist
\item
  Take the following list. Why is its size somewhat misleading?
\end{enumerate}

\begin{Shaded}
\begin{Highlighting}[]
\NormalTok{funs }\OtherTok{\textless{}{-}} \FunctionTok{list}\NormalTok{(mean, sd, var)}
\FunctionTok{obj\_size}\NormalTok{(funs)}
\end{Highlighting}
\end{Shaded}

\begin{verbatim}
## 17.55 kB
\end{verbatim}

\begin{Shaded}
\begin{Highlighting}[]
\FunctionTok{obj\_size}\NormalTok{(funs[[}\DecValTok{1}\NormalTok{]])}
\end{Highlighting}
\end{Shaded}

\begin{verbatim}
## 1.13 kB
\end{verbatim}

\begin{Shaded}
\begin{Highlighting}[]
\FunctionTok{obj\_size}\NormalTok{(funs[[}\DecValTok{2}\NormalTok{]])}
\end{Highlighting}
\end{Shaded}

\begin{verbatim}
## 4.48 kB
\end{verbatim}

\begin{Shaded}
\begin{Highlighting}[]
\FunctionTok{obj\_size}\NormalTok{(funs[[}\DecValTok{3}\NormalTok{]])}
\end{Highlighting}
\end{Shaded}

\begin{verbatim}
## 12.47 kB
\end{verbatim}

It is a list of three functions and tells you the total size, not the
size of each function. These are also base R functions, so they are
already loaded on boot

\begin{enumerate}
\def\labelenumi{\arabic{enumi}.}
\setcounter{enumi}{2}
\tightlist
\item
  Predict the output of the following code:
\end{enumerate}

\begin{Shaded}
\begin{Highlighting}[]
\NormalTok{a }\OtherTok{\textless{}{-}} \FunctionTok{runif}\NormalTok{(}\FloatTok{1e6}\NormalTok{)}
\FunctionTok{obj\_size}\NormalTok{(a)}
\end{Highlighting}
\end{Shaded}

\begin{verbatim}
## 8.00 MB
\end{verbatim}

\begin{Shaded}
\begin{Highlighting}[]
\NormalTok{b }\OtherTok{\textless{}{-}} \FunctionTok{list}\NormalTok{(a, a)}
\FunctionTok{obj\_size}\NormalTok{(b)}
\end{Highlighting}
\end{Shaded}

\begin{verbatim}
## 8.00 MB
\end{verbatim}

\begin{Shaded}
\begin{Highlighting}[]
\FunctionTok{obj\_size}\NormalTok{(a, b)}
\end{Highlighting}
\end{Shaded}

\begin{verbatim}
## 8.00 MB
\end{verbatim}

\begin{Shaded}
\begin{Highlighting}[]
\NormalTok{b[[}\DecValTok{1}\NormalTok{]][[}\DecValTok{1}\NormalTok{]] }\OtherTok{\textless{}{-}} \DecValTok{10}
\FunctionTok{obj\_size}\NormalTok{(b)}
\end{Highlighting}
\end{Shaded}

\begin{verbatim}
## 16.00 MB
\end{verbatim}

\begin{Shaded}
\begin{Highlighting}[]
\FunctionTok{obj\_size}\NormalTok{(a, b)}
\end{Highlighting}
\end{Shaded}

\begin{verbatim}
## 16.00 MB
\end{verbatim}

\begin{Shaded}
\begin{Highlighting}[]
\NormalTok{b[[}\DecValTok{2}\NormalTok{]][[}\DecValTok{1}\NormalTok{]] }\OtherTok{\textless{}{-}} \DecValTok{10}
\FunctionTok{obj\_size}\NormalTok{(b)}
\end{Highlighting}
\end{Shaded}

\begin{verbatim}
## 16.00 MB
\end{verbatim}

\begin{Shaded}
\begin{Highlighting}[]
\FunctionTok{obj\_size}\NormalTok{(a, b)}
\end{Highlighting}
\end{Shaded}

\begin{verbatim}
## 24.00 MB
\end{verbatim}

2.5.3 Exercises

\begin{enumerate}
\def\labelenumi{\arabic{enumi}.}
\tightlist
\item
  Explain why the following code doesn't create a circular list.
\end{enumerate}

\begin{Shaded}
\begin{Highlighting}[]
\NormalTok{x }\OtherTok{\textless{}{-}} \FunctionTok{list}\NormalTok{()}
\NormalTok{x[[}\DecValTok{1}\NormalTok{]] }\OtherTok{\textless{}{-}}\NormalTok{ x}
\end{Highlighting}
\end{Shaded}

Because the list originally contained nothing

\begin{enumerate}
\def\labelenumi{\arabic{enumi}.}
\setcounter{enumi}{1}
\item
  Wrap the two methods for subtracting medians into two functions, then
  use the `bench' package to carefully compare their speeds. How does
  performance change as the number of columns increase?
\item
  What happens if you attempt to use \texttt{tracemem()} on an
  environment? Environments are always modified in place
\end{enumerate}

\end{document}
